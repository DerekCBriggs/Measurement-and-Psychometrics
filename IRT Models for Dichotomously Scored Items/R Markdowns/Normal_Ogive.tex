\documentclass[12pt,]{article}
\usepackage{lmodern}
\usepackage{amssymb,amsmath}
\usepackage{ifxetex,ifluatex}
\usepackage{fixltx2e} % provides \textsubscript
\ifnum 0\ifxetex 1\fi\ifluatex 1\fi=0 % if pdftex
  \usepackage[T1]{fontenc}
  \usepackage[utf8]{inputenc}
\else % if luatex or xelatex
  \ifxetex
    \usepackage{mathspec}
  \else
    \usepackage{fontspec}
  \fi
  \defaultfontfeatures{Ligatures=TeX,Scale=MatchLowercase}
\fi
% use upquote if available, for straight quotes in verbatim environments
\IfFileExists{upquote.sty}{\usepackage{upquote}}{}
% use microtype if available
\IfFileExists{microtype.sty}{%
\usepackage{microtype}
\UseMicrotypeSet[protrusion]{basicmath} % disable protrusion for tt fonts
}{}
\usepackage[margin=1in]{geometry}
\usepackage{hyperref}
\hypersetup{unicode=true,
            pdftitle={The Normal Ogive},
            pdfauthor={Derek Briggs},
            pdfborder={0 0 0},
            breaklinks=true}
\urlstyle{same}  % don't use monospace font for urls
\usepackage{graphicx,grffile}
\makeatletter
\def\maxwidth{\ifdim\Gin@nat@width>\linewidth\linewidth\else\Gin@nat@width\fi}
\def\maxheight{\ifdim\Gin@nat@height>\textheight\textheight\else\Gin@nat@height\fi}
\makeatother
% Scale images if necessary, so that they will not overflow the page
% margins by default, and it is still possible to overwrite the defaults
% using explicit options in \includegraphics[width, height, ...]{}
\setkeys{Gin}{width=\maxwidth,height=\maxheight,keepaspectratio}
\IfFileExists{parskip.sty}{%
\usepackage{parskip}
}{% else
\setlength{\parindent}{0pt}
\setlength{\parskip}{6pt plus 2pt minus 1pt}
}
\setlength{\emergencystretch}{3em}  % prevent overfull lines
\providecommand{\tightlist}{%
  \setlength{\itemsep}{0pt}\setlength{\parskip}{0pt}}
\setcounter{secnumdepth}{0}
% Redefines (sub)paragraphs to behave more like sections
\ifx\paragraph\undefined\else
\let\oldparagraph\paragraph
\renewcommand{\paragraph}[1]{\oldparagraph{#1}\mbox{}}
\fi
\ifx\subparagraph\undefined\else
\let\oldsubparagraph\subparagraph
\renewcommand{\subparagraph}[1]{\oldsubparagraph{#1}\mbox{}}
\fi

%%% Use protect on footnotes to avoid problems with footnotes in titles
\let\rmarkdownfootnote\footnote%
\def\footnote{\protect\rmarkdownfootnote}

%%% Change title format to be more compact
\usepackage{titling}

% Create subtitle command for use in maketitle
\providecommand{\subtitle}[1]{
  \posttitle{
    \begin{center}\large#1\end{center}
    }
}

\setlength{\droptitle}{-2em}

  \title{The Normal Ogive}
    \pretitle{\vspace{\droptitle}\centering\huge}
  \posttitle{\par}
    \author{Derek Briggs}
    \preauthor{\centering\large\emph}
  \postauthor{\par}
      \predate{\centering\large\emph}
  \postdate{\par}
    \date{2/1/2020}


\begin{document}
\maketitle

{
\setcounter{tocdepth}{2}
\tableofcontents
}
\subsection{Background}\label{background}

\subsubsection{Random Variables and Probability
Functions}\label{random-variables-and-probability-functions}

A random variable can be somehwat loosely defined as a quantity that can
have more than one realized value such that the possible values can be
assigned to a probability function.

The convention is to denote Random variables using upper case (usually
italicized) Roman letters. Examples: \(X\), \(Y\), \(Z\). The realized
or observed values of random variables are denoted using lower case
(usually italicized) Roman letters. Example: \(x\), \(y\), \(z\).
Putting the two together, if we write \(P(X=x)=.5\), this expression
says, the probability that random variable \(X\) is equal to the value
\(x\) is \(.5\).

There are two kinds of random variables, those that are \emph{discrete},
and those that are \emph{continuous}. A probability function provides
information about the distribution of a given random variable. When the
variable is discrete, we call this function a probability
\emph{distribution} function. When the variable is continuous, we call
this a probability \emph{density} function. In either case, the
shorthand acronym for this is \textbf{pdf}.

A pdf can be used to tell us the probability of observing a specific
value when the variable is discrete, or values within some defined range
when the variable is continuous.

In contrast, a cumulative distribution function, \textbf{cdf} for short,
means the same thing whether we are dealing with a discrete or
continuous. It tells us \(P(X\leq x)\).

\paragraph{pdf of a Discrete Random
Variable}\label{pdf-of-a-discrete-random-variable}

\[p(x)=P(X=x)\]\\
where\\
1. \(p(x)\geq 0\) (the probability needs to be greater than 0)\\
2. \(\sum_{x} p(x)=1\) (the probabilities of all discrete values must
sum to 1)

\paragraph{cdf of a Discrete Random
Variable}\label{cdf-of-a-discrete-random-variable}

\(F(x)=P(X\leq x)=\sum_{t|t\leq x} p(t)=1\)

\paragraph{pdf of a Continuous Random
Variable}\label{pdf-of-a-continuous-random-variable}

These are a bit more complicated. The Khan Academy actually has a nice
\href{http://www.youtube.com/watch?v=Fvi9A_tEmXQ}{video} I would
recommend.

We represent the pdf as a function \(f(x)\)\\
where\\
\[P(a\leq X \leq b) = \int_{a}^{b}f(x) \; dx\] 1. \(a, b\) are real
numbers and \(a\leq b\)\\
2. \(f(x)\geq 0\) for \(-\infty < 0 < +\infty\)\\
3. \(\int_{-\infty}^{ +\infty} f(x) \; dx=1\)\\
4. \(P(X=c)=0\)

\paragraph{cdf of a Continuous Random
Variable}\label{cdf-of-a-continuous-random-variable}

\[F(x)=P(X\leq x)=\int_{-\infty}^{x}f(t) \; dt \]

\paragraph{From cdf to pdf}\label{from-cdf-to-pdf}

\begin{itemize}
\tightlist
\item
  As is evident from the last expression, to go from a pdf to a cdf for
  a continuous random variable we need to take a definite integral.\\
\item
  To go in the other direction, from a cdf to a pdf for a continuous
  random variable take the first derivative of the cdf with respect to
  \(x\)
\end{itemize}

\[f(x)=\frac{dF(x)}{dx}\]

\subsubsection{Possible Examples of Random
Variables}\label{possible-examples-of-random-variables}

\textbf{Discrete}\\
* The flip of a coin * The color of an M\&M I pull out of in a newly
purchased bag.\\
* The answer of a respondent to a multiple-choice test item.\\
* The answer of a respondent to a Likert-style survey item.\\
\textbf{Continuous}\\
* Age, Height, Weight\\
* Latent Ability \((\theta)\)

What makes these variables random? The belief that the observed values
are governed by a chance process.

\subsubsection{Expected Value of a Random
Variable}\label{expected-value-of-a-random-variable}

If \(X\) is a discrete random variable with pdf \(p(x)\)

\[E(X)=\mu=\sum_{x} x \; p(x)\] If \(X\) is a continuous random variable
with pdf \(f(x)\)

\[E(X)=\mu=\int_{-\infty}^{x}x\; f(x) \; dx\]

\subsubsection{Variance of a Random
Variable}\label{variance-of-a-random-variable}

\[
\begin{align}
\sigma^2 &  = E(X-E(X))^2 \\
         &  = E(X-\mu)^2
\end{align}
\] \#\# The Normal (Gaussian) Distribution

The pdf of the normal distribution is\\
\[f(x)=\frac{1}{\sqrt{2\pi}\sigma}exp\left({\frac{-(x-\mu)^2}{2\sigma^2}}\right)\]
The get the cdf, we need to find a specific area of the normal pdf. We
do this by taking a definite integral\\
\[
\begin{align}
P(X\leq x)=F(x) &=\int_{-\infty}^{x} f(t) \; dt \\
                &=\frac{1}{\sqrt{2\pi}\sigma}\int_{-\infty}^{x} exp\left({\frac{-(t-\mu)^2}{2\sigma^2}}\right) \; dt  
\end{align}  
\]


\end{document}
